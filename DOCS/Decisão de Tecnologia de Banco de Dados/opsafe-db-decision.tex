\documentclass[12pt,a4paper]{article}

% ---- Packages ----
\usepackage[utf8]{inputenc}
\usepackage[brazil]{babel}
\usepackage{geometry}
\usepackage{setspace}
\usepackage{titlesec}
\usepackage{hyperref}
\usepackage{longtable}

% ---- Page setup ----
\geometry{margin=2.5cm}
\setstretch{1.25}

% ---- Title formatting ----
\titleformat{\section}{\Large\bfseries}{\thesection}{1em}{}
\titleformat{\subsection}{\large\bfseries}{\thesubsection}{1em}{}

\begin{document}

% ======================= CAPA =======================
\begin{titlepage}
    \centering
    {\Huge \textbf{OpSafe -- Decisão Arquitetural de Banco de Dados}}\\[1.5cm]
    {\Large \textbf{Versão 1.0}}\\[0.5cm]
    {\large Documento Técnico Empresarial}\\[1cm]
    {\large Autor: Caio César Ponte}\\[0.3cm]
    {\large \textbf{Data: 15 de novembro de 2025}}\\[4cm]

    \vfill
    {\small Este documento formaliza a decisão técnica sobre a tecnologia de banco de dados a ser utilizada no projeto OpSafe, apresentando o contexto, alternativas avaliadas, análise comparativa, riscos, justificativas e decisão final.}
\end{titlepage}

\tableofcontents
\newpage

% ======================= 1. CONTEXTO =======================
\section{Contexto Geral}

O projeto OpSafe é uma plataforma SaaS para gestão operacional de empresas de segurança privada e patrimonial.  
Seu escopo envolve:

\begin{itemize}
    \item Controle de inventário de equipamentos (radios, EPIs, coletes, viaturas);
    \item Rastreamento de movimentações (checkout, checkin, transferência);
    \item Emissão e auditoria de termos digitais eletrônicos;
    \item Gestão de operadores e postos de serviço;
    \item Histórico completo de manutenção preventiva e corretiva;
    \item Alertas automáticos e trilhas de auditoria;
    \item Multi-tenant com isolamento por organização.
\end{itemize}

O banco de dados precisa atender:

\begin{itemize}
    \item Alta flexibilidade nos documentos e evolução das regras de negócio;
    \item Modelo rico de consultas transacionais;
    \item Capacidade de suportar relatórios operacionais (agregações);
    \item Multi-tenant baseado em \texttt{organizationId};
    \item Baixo overhead de manutenção;
    \item Suporte a auditoria histórica extensa;
    \item Escalabilidade horizontal simples.
\end{itemize}

O objetivo deste documento é registrar formalmente a escolha tecnológica do banco de dados primário da plataforma.

% ======================= 2. REQUISITOS =======================
\section{Requisitos Técnicos e de Negócio}

\subsection{2.1 Requisitos Funcionais do Banco}

\begin{itemize}
    \item Suporte a documentos complexos com subestruturas;
    \item Facilidade na criação e evolução de coleções;
    \item Consultas filtradas por múltiplos campos;
    \item Agregações para relatórios (sum, group, project);
    \item Capacidade de trabalhar com auditoria (\texttt{before/after});
    \item Operações de soft delete com filtros globais;
    \item Consultas multi-tenant eficientes.
\end{itemize}

\subsection{2.2 Requisitos Não Funcionais}

\begin{itemize}
    \item Confiabilidade e redundância;
    \item Escalabilidade em nuvem;
    \item Baixo acoplamento ao backend;
    \item Ferramentas maduras de administração;
    \item Drivers estáveis em TypeScript;
    \item Baixo custo operacional inicial.
\end{itemize}

% ======================= 3. ALTERNATIVAS =======================
\section{Alternativas Avaliadas}

Foram consideradas as seguintes tecnologias NoSQL viáveis:

\begin{itemize}
    \item \textbf{MongoDB Atlas}
    \item \textbf{Google Firestore}
    \item \textbf{AWS DynamoDB}
    \item \textbf{PostgreSQL com JSONB (alternativa híbrida)}
\end{itemize}

% ======================= 4. ANÁLISE COMPARATIVA =======================
\section{Análise Comparativa}

\subsection{4.1 Critérios Usados}

\begin{itemize}
    \item Flexibilidade de modelo
    \item Complexidade de consultas
    \item Manutenção e custo operacional
    \item Curva de aprendizado
    \item Suporte a multi-tenant
    \item Adequação ao fluxo de auditoria
    \item Recursos para agregações e relatórios
\end{itemize}

\subsection{4.2 Tabela Comparativa}

\begin{longtable}{|p{4cm}|p{3cm}|p{3cm}|p{3cm}|}
\hline
\textbf{Critério} & \textbf{MongoDB} & \textbf{Firestore} & \textbf{DynamoDB} \\
\hline
Modelo de documentos & Excelente & Bom & Bom \\
\hline
Consultas complexas & Excelente & Limitado & Limitado \\
\hline
Agregações / Relatórios & Forte & Fraco & Ausente \\
\hline
Multi-tenant por ID & Simples & Simples & Requer planejamento rígido \\
\hline
Auditoria (before/after) & Natural & Difícil & Difícil \\
\hline
Curva de aprendizado & Baixa & Baixa & Alta \\
\hline
Escalabilidade & Excelente & Muito alta & Muito alta \\
\hline
Custo mínimo & Moderado & Baixo & Baixo \\
\hline
Flexibilidade para evoluir regras & Excelente & Média & Baixa \\
\hline
\end{longtable}

% ======================= 5. DISCUSSÃO =======================
\section{Discussão}

\subsection{5.1 MongoDB}

Atende de forma superior aos requisitos principais:

\begin{itemize}
    \item Flexibilidade alta para evolução incremental das coleções;
    \item Agregations poderosas para relatórios operacionais;
    \item Suporte natural a objetos aninhados;
    \item Adaptabilidade a documentos grandes, como termos digitais;
    \item Modelo fortíssimo para auditoria (pode armazenar snapshots grandes sem impacto conceitual);
    \item Baixa fricção para multi-tenant;
    \item Mongoose e driver nativo maduros em Node/TypeScript.
\end{itemize}

\subsection{5.2 Firestore}

Excelente para aplicações em tempo real, porém possui:

\begin{itemize}
    \item Limitações severas de query combinada;
    \item Relatórios complexos exigem reestruturações;
    \item Nível baixo de suporte para documentos grandes de auditoria.
\end{itemize}

\subsection{5.3 DynamoDB}

Extremamente escalável, porém:

\begin{itemize}
    \item Requer modelagem antecipada dos padrões de acesso;
    \item Não se adequa ao estágio de exploração e evolução contínua do projeto;
    \item Auditoria complexa devido ao modelo de chave simples.
\end{itemize}

% ======================= 6. DECISÃO =======================
\section{Decisão Final}

Após a análise técnica, o banco de dados escolhido para o OpSafe é:

\begin{center}
    {\Large \textbf{MongoDB Atlas (Nuvem Gerenciada)}}
\end{center}

\subsection{6.1 Justificativas}

\begin{itemize}
    \item É a solução que melhor atende ao modelo multi-tenant do projeto;
    \item Permite evolução rápida das regras de negócio;
    \item Suporta consultas e agregações avançadas necessárias para relatórios;
    \item Lida bem com documentos grandes e históricos de auditoria;
    \item Possui ecossistema maduro para TypeScript;
    \item Reduz complexidade operacional por ser totalmente gerenciado.
\end{itemize}

% ======================= 7. RISCOS =======================
\section{Riscos Identificados}

\begin{itemize}
    \item \textbf{Custo pode crescer} com volume de dados;
    \item \textbf{Indexação incorreta} pode afetar performance;
    \item \textbf{Auditoria extensa} pode gerar coleções muito grandes;
\end{itemize}

\subsection{Mitigações}

\begin{itemize}
    \item Monitoramento e revisão trimestral de índices;
    \item Particionamento lógico por \texttt{organizationId} em consultas intensivas;
    \item Armazenamento de auditorias frias em buckets externos quando necessário;
    \item Uso de backups automáticos e política de retenção.
\end{itemize}

% ======================= 8. RESULTADO E PRÓXIMOS PASSOS =======================
\section{Resultado}

A decisão técnica está formalizada: **MongoDB Atlas será o banco oficial do OpSafe**.  
Esta definição orientará:

\begin{itemize}
    \item O desenvolvimento do backend;
    \item A criação dos repositórios de domínio;
    \item A arquitetura de índices;
    \item Os padrões de auditoria do sistema.
\end{itemize}

\subsection{Próximos passos}

\begin{enumerate}
    \item Criar cluster MongoDB Atlas (dev, stage e prod);
    \item Definir índices iniciais por coleção;
    \item Implementar o primeiro repositório (organizations);
    \item Criar estrutura base da API com filtros globais por \texttt{organizationId} e \texttt{isDeleted}.
\end{enumerate}

\end{document}
