\documentclass[12pt,a4paper]{article}

% Pacotes básicos
\usepackage[utf8]{inputenc}
\usepackage[brazil]{babel}
\usepackage{geometry}
\usepackage{setspace}
\usepackage{titlesec}
\usepackage{hyperref}
\usepackage{lipsum}

% Margens
\geometry{
  left=2.5cm,
  right=2.5cm,
  top=2.5cm,
  bottom=2.5cm
}

% Espaçamento
\onehalfspacing

% Aparência dos títulos
\titleformat{\section}{\bfseries\large}{\thesection}{1em}{}
\titleformat{\subsection}{\bfseries\normalsize}{\thesubsection}{1em}{}

% Informações do documento
\title{\textbf{Documento de Iniciação do Projeto \\ \vspace{0.3cm}
OpSafe – Plataforma de Gestão de Equipamentos Operacionais para Empresas de Segurança Privada}}
\author{Caio César Ponte}
\date{15 de novembro de 2025}

\begin{document}

\maketitle
\thispagestyle{empty}

\newpage

\section*{Resumo Executivo}

O presente documento descreve a concepção inicial do projeto \textbf{OpSafe}, uma plataforma SaaS voltada para empresas de segurança privada, destinada à gestão completa do ciclo de vida de equipamentos operacionais tais como EPIs, rádios comunicadores, viaturas, dispositivos eletrônicos e demais itens essenciais à operação de vigilância.

O projeto parte da necessidade crescente de profissionalização, compliance e rastreabilidade no setor, impulsionada pela nova Lei nº 14.967/2024, que institui o Estatuto da Segurança Privada e da Segurança das Instituições Financeiras. O objetivo é criar uma solução clara, simples e orientada para resultados, permitindo que empresas reduzam perdas, aumentem eficiência e facilitem processos de auditoria.

\newpage

\section{Contexto de Mercado}

O setor de segurança privada é um dos maiores do país em número de trabalhadores e volume de contratos, com quase cinco mil empresas autorizadas a operar pela Polícia Federal. Essas empresas atuam em áreas como vigilância patrimonial, monitoramento eletrônico, segurança de eventos, segurança em instituições financeiras, escolta armada e transporte de valores.

Frente ao aumento da demanda e da complexidade operacional, cresce também a necessidade de controle rigoroso de equipamentos de uso contínuo, sob risco financeiro, trabalhista e regulatório.

Custos como extravio, danos não documentados, EPIs vencidos, manutenção irregular de viaturas e ausência de inventário atualizado prejudicam a operação e reduzirem competitividade.

\section{Contexto Regulatório Atualizado}

A regulação do setor de segurança privada no Brasil passou por uma transformação significativa com a publicação da \textbf{Lei nº 14.967/2024}, sancionada em 9 de setembro de 2024, que revogou integralmente a antiga Lei nº 7.102/1983. A nova lei moderniza e organiza o setor, trazendo definições mais claras sobre tipos de serviços, requisitos de funcionamento e responsabilidades das empresas.

Entre os pontos mais relevantes para este projeto, destacam-se:

\begin{itemize}
    \item A prestação de serviços de segurança privada depende de autorização prévia da Polícia Federal.
    \item A lei define serviços como vigilância patrimonial, segurança eletrônica, escolta armada, segurança de eventos e transporte de valores.
    \item Estabelece capital social mínimo para funcionamento de cada categoria de serviço.
    \item Exige comprovação de garantias financeiras e capacidade operacional.
    \item Reforça a necessidade de registro e controle de equipamentos e pessoal, incluindo obrigações de prestação de contas e compliance.
    \item Prevê revisões periódicas de critérios e valores por regulamento futuro da Polícia Federal.
\end{itemize}

Tais fatores regulatórios reforçam a necessidade de ferramentas que auxiliem empresas a comprovar conformidade e manter controles transparentes e atualizados.

\section{Problema Central}

Empresas de segurança privada enfrentam desafios recorrentes na gestão de equipamentos operacionais. Os principais problemas identificados são:

\begin{itemize}
    \item Controle manual e descentralizado de equipamentos essenciais.
    \item Dificuldade em saber quem está com qual item durante os turnos.
    \item Falta de rastreabilidade histórica por colaborador, posto ou contrato.
    \item Manutenções preventivas inexistentes ou mal registradas.
    \item EPIs vencidos, especialmente coletes balísticos, sem alertas ou controle.
    \item Inventários demorados e sujeitos a erro humano.
    \item Ausência de relatórios estruturados para auditorias e renovações junto à Polícia Federal.
\end{itemize}

A soma desses pontos gera impacto financeiro, operacional e regulatório, criando terreno adequado para o surgimento de uma solução especializada.

\section{Objetivos do Projeto}

\subsection{Objetivo Geral}
Desenvolver uma plataforma SaaS simples, robusta e eficiente para controle integral do ciclo de vida de equipamentos em empresas de segurança privada, com foco em rastreabilidade, conformidade regulatória e otimização operacional.

\subsection{Objetivos Específicos}

\begin{enumerate}
    \item Reduzir perdas e extravios de equipamentos.
    \item Aumentar o controle e a rastreabilidade de itens em uso.
    \item Diminuir tempo gasto com inventários e planilhas manuais.
    \item Fornecer relatórios claros para processos de auditoria da Polícia Federal e contratos privados.
    \item Criar base de dados confiável para manutenção preventiva de ativos.
\end{enumerate}

\section{Escopo Inicial – MVP}

O MVP inicial da plataforma OpSafe deverá contemplar:

\begin{itemize}
    \item Cadastro de equipamentos com identificador único.
    \item Cadastro de colaboradores, postos e clientes.
    \item Registro de entrega e devolução por meio de \textbf{Termo Digital}.
    \item Painel de visão rápida indicando quem está com qual equipamento.
    \item Controle de validade de EPIs e alertas para substituição.
    \item Controle de manutenção de viaturas e rádios.
    \item Relatórios básicos em PDF e Excel.
\end{itemize}

\section{Personas}

\subsection{Supervisor Operacional}
Responsável pela gestão de equipes em campo, acompanhando turnos e postos. Precisa de rapidez e clareza na identificação da situação de cada equipamento.

\subsection{Gestor de Equipamentos / Frota}
Cuida de ativos caros e críticos, como viaturas e rádios. Depende de histórico detalhado para justificar custos e garantir continuidade operacional.

\subsection{Diretor / Proprietário}
Busca estabilidade regulatória, redução de custos e visibilidade real da operação. Enxerga a plataforma como ferramenta de gestão e compliance.

\section{Regras de Negócio}

\subsection{Identificação única de cada equipamento}
Cada item é registrado com identificador único, permitindo rastreabilidade histórica.

\subsection{Termo Digital obrigatório para entrega e devolução}
Operações de entrega e devolução devem gerar registro digital com data, responsável e validade.

\subsection{EPIs com validade vencida não podem ser alocados}
A plataforma bloqueia automaticamente a entrega de EPIs vencidos, em alinhamento a boas práticas de segurança do trabalho.

\subsection{Manutenções obrigam abertura de ordem registrada}
Toda manutenção preventiva ou corretiva deve ser registrada e vinculada ao ativo.

\subsection{Regras específicas por contrato}
Cada contrato de prestação de serviços pode definir equipamentos mínimos e exigências próprias.

\section{Arquitetura de Módulos – Visão Funcional}

\subsection{Cadastro}
Equipamentos, colaboradores, clientes, categorias e tipos.

\subsection{Operação}
Entrega e devolução, painel de uso, turnos e alertas.

\subsection{Manutenção}
Ordens de manutenção, custos, prazos, retorno de itens.

\subsection{Relatórios}
Inventário, extravios, conformidade contratual, manutenção e validade de EPIs.

\section{Fluxos Essenciais}

\subsection{Fluxo de Cadastro de Novo Equipamento}
Informações básicas, estado inicial como ``Disponível'' e capacidade de rastreio imediato.

\subsection{Fluxo de Entrega ao Colaborador}
Seleção do item, verificação automática de validade e emissão do Termo Digital.

\subsection{Fluxo de Devolução}
Registro de retorno do item, mudança de estado e abertura de manutenção quando necessário.

\section{Métricas de Sucesso}

\begin{itemize}
    \item Redução da taxa de extravio por colaborador.
    \item Redução do tempo médio de inventário mensal.
    \item Aumento da taxa de EPIs válidos.
    \item Quantidade de manutenções preventivas realizadas dentro do prazo.
\end{itemize}

\section{Premissas e Riscos}

\subsection{Premissas}
As empresas médias do setor estão dispostas a adotar ferramentas simples e de implantação rápida.

\subsection{Riscos}
Resistência cultural ao uso de sistemas por supervisores de campo; ausência de integrações pode limitar valor inicial.

\section{Roadmap Macro}

\subsection{Fase 0 – Validação}
Entrevistas com empresas do Nordeste para confirmar hipóteses e ajustar diretrizes.

\subsection{Fase 1 – MVP}
Construção do núcleo do sistema e implantação em empresa piloto.

\subsection{Fase 2 – Consolidação}
Aprimoramento da experiência e introdução de novos módulos.

\subsection{Fase 3 – Expansão Regional}
Escalonamento para múltiplas empresas de médio porte.

\section*{Conclusão}

O projeto OpSafe nasce com o objetivo de modernizar a gestão de equipamentos operacionais no setor de segurança privada, alinhado às exigências regulatórias mais recentes e às necessidades reais de eficiência e controle das empresas.

Este documento serve como marco inicial estruturado para desenvolvimento, validação e expansão futura do produto.

\end{document}
