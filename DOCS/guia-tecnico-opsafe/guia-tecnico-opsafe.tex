\documentclass[12pt,a4paper]{article}

\usepackage[utf8]{inputenc}
\usepackage[brazil]{babel}
\usepackage{geometry}
\usepackage{setspace}
\usepackage{titlesec}
\usepackage{hyperref}
\usepackage{xcolor}

\geometry{
  left=2.5cm,
  right=2.5cm,
  top=2.5cm,
  bottom=2.5cm
}

\onehalfspacing

\titleformat{\section}{\bfseries\large}{\thesection}{1em}{}
\titleformat{\subsection}{\bfseries\normalsize}{\thesubsection}{1em}{}

\title{\textbf{Guia Técnico do Projeto OpSafe \\ \vspace{0.3cm}
Diretrizes Técnicas, Arquiteturais, Padrões de Qualidade e Segurança}}
\author{Caio César Ponte}
\date{15 de novembro de 2025}

\begin{document}

\maketitle
\thispagestyle{empty}
\newpage

% ============================
\section*{Introdução}
Este documento técnico define os padrões, tecnologias, métodos de engenharia, regras de qualidade e diretrizes arquiteturais que serão seguidos no projeto \textbf{OpSafe}. Ele serve como guia para desenvolvimento, validação, versionamento, governança, testes, segurança, LGPD e boas práticas para manter a plataforma escalável, segura e consistente.

\newpage

% ============================
\section{Padrões de Interface}

\subsection{Família de Ícones}
A plataforma utilizará a biblioteca \textbf{Lucide Icons} como padrão visual principal.  
Características:
\begin{itemize}
  \item Traços lineares, leves e minimalistas.
  \item Conjunto amplo e padronizado.
  \item Adaptado a dark/light mode.
\end{itemize}

\subsection{Família de Cores}
A identidade visual seguirá um conjunto modular:

\begin{itemize}
  \item \textbf{Primária}: \#1A73E8 (azul operacional)
  \item \textbf{Secundária}: \#0A2F5A (azul profundo – seriedade e segurança)
  \item \textbf{Atenção}: \#F9A825
  \item \textbf{Sucesso}: \#43A047
  \item \textbf{Erro}: \#E53935
  \item \textbf{Cinzas}: escala de \#111111 a \#F5F5F5
\end{itemize}

\subsection{Família de Fontes}
\begin{itemize}
  \item \textbf{Inter} (plataforma Web)
  \item \textbf{Roboto} (mobile e compatibilidade Android)
  \item \textbf{SF Pro Text} (compatibilidade iOS)
\end{itemize}

\subsection{Responsividade e Grid}
A aplicação segue:
\begin{itemize}
  \item \textbf{CSS Grid} + \textbf{Flexbox}
  \item Breakpoints: 480px, 768px, 1024px, 1440px
  \item Suporte completo ao Mobile-first
\end{itemize}

\subsection{Layout}
\begin{itemize}
  \item \textbf{Layout Master-Detail} para equipamentos.
  \item \textbf{Cards modulares} para dashboards.
  \item \textbf{Side Navigation em modo desktop}.
  \item \textbf{Bottom Navigation em mobile}.
\end{itemize}

\newpage

% ============================
\section{Arquitetura de Software}

\subsection{Clean Architecture – Camadas}
\begin{itemize}
  \item \textbf{Domain}: Regras de negócio puras, entidades.
  \item \textbf{Use Cases}: Operações atômicas da aplicação.
  \item \textbf{Infrastructure}: Database, API, drivers externos.
  \item \textbf{Presentation}: Camada de UI (React / Mobile).
\end{itemize}

\subsection{SOLID}
Aplicado em todos os módulos:
\begin{itemize}
  \item SRP: Um módulo = uma responsabilidade.
  \item OCP: Código aberto para extensão, fechado para modificação.
  \item LSP: Módulos substituíveis sem quebrar fluxos.
  \item ISP: Interfaces pequenas, específicas.
  \item DIP: Camadas superiores dependem de abstrações, não implementações.
\end{itemize}

\subsection{Padrões de Código}
\begin{itemize}
  \item TypeScript estrito.
  \item ESLint + Prettier.
  \item Commits semânticos (feat/fix/docs/refactor/test).
  \item Versionamento semantic versioning (semver).
\end{itemize}

\newpage

% ============================
\section{Banco de Dados}

\subsection{Modelo NoSQL – Padrão}
O OpSafe utilizará um banco \textbf{MongoDB} ou \textbf{Firestore} com os princípios:

\begin{itemize}
  \item Coleções independentes.
  \item Documentos auto-contidos.
  \item Campos versionados para retrocompatibilidade.
\end{itemize}

\subsection{Estrutura Base}
\begin{itemize}
  \item \texttt{equipments}
  \item \texttt{operators}
  \item \texttt{assignments}
  \item \texttt{maintenance}
  \item \texttt{contracts}
  \item \texttt{alerts}
  \item \texttt{audit\_logs}
\end{itemize}

\subsection{Blockchain (Opcional e Modular)}
Aplicável para:
\begin{itemize}
  \item garantias de integridade de termos digitais;
  \item trilhas de auditoria imutáveis;
\end{itemize}
Usando:
\begin{itemize}
  \item Hash SHA-256;
  \item Armazenamento fora da cadeia (off-chain) para dados pesados.
\end{itemize}

\newpage

% ============================
\section{Segurança da Informação}

\subsection{LGPD – Conformidade}
\begin{itemize}
  \item Minimização de dados.
  \item Consentimento explícito para operadores.
  \item Relatórios de transparência.
  \item Controlador e operador bem definidos.
\end{itemize}

\subsection{Criptografia}
\begin{itemize}
  \item AES-256 para dados sensíveis.
  \item TLS 1.3 para tráfego.
  \item Hashing Argon2 para senhas.
\end{itemize}

\subsection{Autenticação e Autorização}
\begin{itemize}
  \item OAuth2 + JWT.
  \item Expiração de tokens + refresh tokens.
  \item RBAC (Role Based Access Control).
\end{itemize}

\subsection{DevSecOps}
\begin{itemize}
  \item SAST em cada PR.
  \item Dependabot para pacotes.
  \item Pipeline com bloqueio de CVEs acima de CVSS 7.
\end{itemize}

\newpage

% ============================
\section{Métodos e Fluxos de Cadastro}

\subsection{Cadastro de Equipamento}
\begin{itemize}
  \item Identificador único.
  \item Categoria e contrato associado.
  \item Ciclo de vida inicial como ``Disponível''.
\end{itemize}

\subsection{Cadastro de Operador}
\begin{itemize}
  \item Dados mínimos (LGPD).
  \item Função e posto.
  \item Termo de responsabilidade digital.
\end{itemize}

\subsection{Cadastro de Manutenção}
\begin{itemize}
  \item Preventiva ou corretiva.
  \item Data programada e realizada.
  \item Registro de custo.
\end{itemize}

\newpage

% ============================
\section{Testes e Qualidade}

\subsection{Testes Unitários}
\begin{itemize}
  \item Cobertura mínima: 85\%.
  \item Testes para cada use case isolado.
\end{itemize}

\subsection{Testes de Integração}
\begin{itemize}
  \item Interação entre camadas.
  \item Testes com banco real (ambiente isolado).
\end{itemize}

\subsection{Testes End-to-End}
\begin{itemize}
  \item Cypress / Playwright.
  \item Simular fluxo completo: entrega → termo digital → devolução.
\end{itemize}

\subsection{Ambiente Controlado}
\begin{itemize}
  \item DEV → STAGE → PROD
  \item STAGE com dados fictícios.
  \item Proibição de dados reais fora do PROD.
\end{itemize}

\newpage

% ============================
\section{CI/CD – Entrega Contínua}

\subsection{Pipeline}
\begin{itemize}
  \item Build
  \item Testes unitários
  \item SAST
  \item Deploy automático para STAGE
  \item Deploy manual e aprovado para PROD
\end{itemize}

\subsection{Logs e Observabilidade}
\begin{itemize}
  \item Logs estruturados (JSON)
  \item Monitoramento via Prometheus ou Datadog
  \item Alertas de uptime
\end{itemize}

\newpage

% ============================
\section{Regras de Negócio – Padronização}

\subsection{Identificação Única}
Nenhum recurso pode existir sem ID único global.

\subsection{Termos Digitais}
Toda entrega e devolução gera registro permanente auditável.

\subsection{EPIs Vencidos}
Sistema bloqueia automaticamente distribuição.

\subsection{Manutenção Obrigatória}
Itens com defeito entram em quarentena até liberação.

\newpage

% ============================
\section{Definição de Pronto (DoD)}
Um item de trabalho só é considerado concluído quando:

\begin{itemize}
  \item Código revisado.
  \item Testes unitários e integração passando.
  \item Documentação atualizada.
  \item Sem falhas de lint.
  \item Sem vulnerabilidades críticas.
\end{itemize}

\newpage

% ============================
\section{Conclusão}

Este documento estabelece a base técnica, arquitetural e operacional que sustenta o projeto OpSafe. Ele deve ser utilizado como referência contínua pelo time de desenvolvimento, design, análise, segurança e gestão, garantindo consistência, escalabilidade e aderência à legislação aplicável, especialmente à Lei nº 14.967/2024 e à LGPD.

\end{document}
