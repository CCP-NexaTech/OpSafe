\documentclass[12pt,a4paper]{article}

% ---- Packages ----
\usepackage[utf8]{inputenc}
\usepackage[brazil]{babel}
\usepackage{geometry}
\usepackage{setspace}
\usepackage{titlesec}
\usepackage{hyperref}
\usepackage{longtable}
\usepackage{array}

% ---- Page setup ----
\geometry{margin=2.5cm}
\setstretch{1.25}

% ---- Title format ----
\titleformat{\section}{\large\bfseries}{\thesection}{1em}{}
\titleformat{\subsection}{\normalsize\bfseries}{\thesubsection}{1em}{}

\begin{document}

% ================== CAPA ==================
\begin{titlepage}
    \centering
    {\Huge \textbf{OpSafe -- Data Model Specification}}\\[1.5cm]
    {\Large \textbf{Versão v0.1}}\\[0.5cm]
    {\large Documento Técnico Empresarial}\\[1cm]
    {\large Caio César Ponte}\\[0.3cm]
    {\large \textbf{Data: 15 de novembro de 2025}}\\[4cm]

    \vfill
    {\small Este documento descreve o modelo de dados inicial da plataforma OpSafe, incluindo estruturas, padrões de auditoria, regras de negócio, requisitos de segurança e aderência à LGPD.}
\end{titlepage}

\tableofcontents
\newpage

% ================== 1. OBJETIVO ==================
\section{Objetivo do Documento}

Este documento define o \textbf{Modelo de Dados v0.1} do projeto OpSafe, cobrindo:

\begin{itemize}
    \item Estruturas de coleções (NoSQL) e seus campos;
    \item Regras e relacionamentos principais;
    \item Padrões globais de auditoria e soft delete;
    \item Marcação conceitual de dados pessoais (LGPD);
    \item Base para especificações de API, testes e migrações futuras.
\end{itemize}

Serve como referência única para desenvolvedores, arquitetos, QA e ferramentas de IA que darão suporte ao projeto.

% ================== 2. PRINCÍPIOS ==================
\section{Princípios Gerais}

\subsection{Multi-tenant}

\begin{itemize}
    \item Todas as coleções de negócio incluem \texttt{organizationId}.
    \item Cada \texttt{organization} representa uma empresa cliente distinta.
    \item Não há compartilhamento lógico de dados entre organizações diferentes.
\end{itemize}

\subsection{Modelo NoSQL}

\begin{itemize}
    \item Banco orientado a documentos (MongoDB, Firestore ou equivalente).
    \item Relacionamentos expressos por IDs (\texttt{equipmentId}, \texttt{operatorId}, etc.).
    \item Preferência por documentos auto-contidos para consultas mais comuns.
\end{itemize}

\subsection{Convenções de Tipos}

\begin{itemize}
    \item \textbf{string} -- texto ou identificadores.
    \item \textbf{number} -- valores numéricos.
    \item \textbf{boolean} -- verdadeiro/falso.
    \item \textbf{datetime} -- data/hora em UTC, formato ISO 8601.
    \item \textbf{enum} -- conjunto finito de valores válidos.
    \item \textbf{object} -- subdocumento com campos próprios.
    \item \textbf{array<T>} -- lista de elementos do tipo \texttt{T}.
\end{itemize}

\subsection{Padrão de Auditoria e Soft Delete}

Todas as coleções de negócio devem conter os campos:

\begin{longtable}{|p{4cm}|p{3cm}|p{7cm}|}
\hline
\textbf{Campo} & \textbf{Tipo} & \textbf{Descrição}\\ \hline
createdAt & datetime & Data/hora de criação do documento \\ \hline
createdByUserId & string/null & ID do usuário que criou (ou null se sistema) \\ \hline
updatedAt & datetime/null & Data/hora da última atualização \\ \hline
updatedByUserId & string/null & ID do usuário que atualizou por último \\ \hline
deletedAt & datetime/null & Data/hora de exclusão lógica (soft delete) \\ \hline
deletedByUserId & string/null & ID do usuário que executou o soft delete \\ \hline
isDeleted & boolean & Indicador de exclusão lógica (padrão: false) \\ \hline
\end{longtable}

Regra padrão:

\begin{itemize}
    \item Se \texttt{isDeleted = true}, então \texttt{deletedAt} não pode ser nulo.
    \item Consultas operacionais devem considerar apenas registros com \texttt{isDeleted = false}, salvo quando explicitamente indicado o contrário.
\end{itemize}

\subsection{LGPD -- Classificação Conceitual}

Internamente, os campos são classificados em:

\begin{itemize}
    \item \textbf{PII}: dado pessoal identificável (nome, e-mail, IP, etc.);
    \item \textbf{Sensitive}: dado pessoal sensível (por exemplo, CPF, se armazenado);
    \item \textbf{Business}: dados puramente de negócio (status de equipamento, ID de contrato, etc.).
\end{itemize}

Esta classificação orienta decisões de:

\begin{itemize}
    \item Minimização de coleta;
    \item Anonimização de dados;
    \item Políticas de retenção;
    \item Resposta a solicitações de titulares (LGPD).
\end{itemize}

% ================== 3. LISTA DE COLEÇÕES ==================
\section{Lista de Coleções do Modelo v0.1}

\begin{enumerate}
    \item organizations
    \item users
    \item operators
    \item clients
    \item contracts
    \item posts
    \item equipmentTypes
    \item equipments
    \item assignments
    \item terms
    \item maintenanceOrders
    \item alerts
    \item auditLogs
    \item customFields (opcional / extensível)
\end{enumerate}

% ================== 4. ESPECIFICAÇÕES ==================
\section{Especificações de Coleções}

% -------- 4.1 organizations --------
\subsection{organizations}

Representa empresas que utilizam o OpSafe (tenant raiz).

\begin{longtable}{|p{4cm}|p{3cm}|p{7cm}|}
\hline
\textbf{Campo} & \textbf{Tipo} & \textbf{Descrição} \\ \hline
id & string & Identificador único da organização \\ \hline
name & string & Nome da organização (razão social ou fantasia) \\ \hline
cnpj & string/null & CNPJ da empresa (Business) \\ \hline
contactEmail & string/null & E-mail geral de contato (PII de contato) \\ \hline
contactPhone & string/null & Telefone geral de contato (PII de contato) \\ \hline
status & enum & \texttt{"active"}, \texttt{"suspended"}, \texttt{"trial"} \\ \hline
createdAt, createdByUserId, updatedAt, updatedByUserId, deletedAt, deletedByUserId, isDeleted & vários & Campos de auditoria e soft delete \\ \hline
\end{longtable}

% -------- 4.2 users --------
\subsection{users}

Usuários internos do sistema (admin, gestor, supervisor, etc.).

\begin{longtable}{|p{4cm}|p{3cm}|p{7cm}|}
\hline
id & string & ID único do usuário \\ \hline
organizationId & string & Organização à qual o usuário pertence \\ \hline
name & string & Nome completo do usuário (PII) \\ \hline
email & string & E-mail de login, único por organização (PII) \\ \hline
role & enum & \texttt{"admin"}, \texttt{"manager"}, \texttt{"supervisor"}, \texttt{"viewer"} \\ \hline
passwordHash & string & Hash seguro da senha (Sensitive, ex.: Argon2) \\ \hline
status & enum & \texttt{"active"}, \texttt{"inactive"} \\ \hline
lastLoginAt & datetime/null & Data/hora do último login \\ \hline
createdAt, createdByUserId, updatedAt, updatedByUserId, deletedAt, deletedByUserId, isDeleted & vários & Auditoria e soft delete \\ \hline
\end{longtable}

% -------- 4.3 operators --------
\subsection{operators}

Operadores que recebem e utilizam equipamentos (vigilantes, motoristas etc.). Podem não ter acesso direto ao sistema.

\begin{longtable}{|p{4cm}|p{3cm}|p{7cm}|}
\hline
id & string & ID único do operador \\ \hline
organizationId & string & Organização associada \\ \hline
fullName & string & Nome completo (PII) \\ \hline
identifierCode & string & Código interno / matrícula (PII/Business) \\ \hline
cpf & string/null & CPF (PII sensível, apenas se estritamente necessário) \\ \hline
role & string & Função (ex.: vigilante, motorista) \\ \hline
status & enum & \texttt{"active"}, \texttt{"inactive"} \\ \hline
contactPhone & string/null & Telefone (PII) \\ \hline
notes & string/null & Observações gerais de negócio \\ \hline
createdAt, createdByUserId, updatedAt, updatedByUserId, deletedAt, deletedByUserId, isDeleted & vários & Auditoria e soft delete \\ \hline
\end{longtable}

% -------- 4.4 clients --------
\subsection{clients}

Clientes finais atendidos pela organização de segurança (condomínios, empresas, indústrias).

\begin{longtable}{|p{4cm}|p{3cm}|p{7cm}|}
\hline
id & string & ID único do cliente \\ \hline
organizationId & string & Organização dona deste cliente \\ \hline
name & string & Nome do cliente \\ \hline
cnpj & string/null & CNPJ do cliente \\ \hline
contactName & string/null & Nome do representante (PII) \\ \hline
contactEmail & string/null & E-mail do representante (PII) \\ \hline
contactPhone & string/null & Telefone do representante (PII) \\ \hline
address & object/null & Endereço: rua, cidade, estado, CEP \\ \hline
createdAt, createdByUserId, updatedAt, updatedByUserId, deletedAt, deletedByUserId, isDeleted & vários & Auditoria e soft delete \\ \hline
\end{longtable}

% -------- 4.5 contracts --------
\subsection{contracts}

Contratos de prestação de serviços com regras mínimas de equipamentos.

\begin{longtable}{|p{4cm}|p{3cm}|p{7cm}|}
\hline
id & string & ID único do contrato \\ \hline
organizationId & string & Organização dona do contrato \\ \hline
clientId & string & Referência ao cliente \\ \hline
code & string & Código interno do contrato \\ \hline
description & string & Descrição resumida \\ \hline
startDate & datetime & Data de início \\ \hline
endDate & datetime/null & Data de término (se houver) \\ \hline
status & enum & \texttt{"draft"}, \texttt{"active"}, \texttt{"terminated"} \\ \hline
minEquipmentRules & array<object> & Regras mínimas por posto/tipo de equipamento \\ \hline
\multicolumn{3}{|p{14cm}|}{
\textbf{Estrutura de minEquipmentRules:}
\begin{itemize}
    \item postId: string/null
    \item equipmentTypeId: string
    \item quantityMin: number
\end{itemize}
} \\ \hline
createdAt, createdByUserId, updatedAt, updatedByUserId, deletedAt, deletedByUserId, isDeleted & vários & Auditoria e soft delete \\ \hline
\end{longtable}

% -------- 4.6 posts --------
\subsection{posts}

Postos de serviço onde há operadores e equipamentos alocados.

\begin{longtable}{|p{4cm}|p{3cm}|p{7cm}|}
\hline
id & string & ID do posto \\ \hline
organizationId & string & Organização dona do posto \\ \hline
clientId & string & Cliente associado \\ \hline
name & string & Nome do posto (ex.: Portaria Principal) \\ \hline
location & string/null & Descrição livre da localização \\ \hline
contractId & string/null & Contrato vinculado \\ \hline
status & enum & \texttt{"active"}, \texttt{"inactive"} \\ \hline
createdAt, createdByUserId, updatedAt, updatedByUserId, deletedAt, deletedByUserId, isDeleted & vários & Auditoria e soft delete \\ \hline
\end{longtable}

% -------- 4.7 equipmentTypes --------
\subsection{equipmentTypes}

Tipos de equipamentos, com características gerais de cada categoria.

\begin{longtable}{|p{4cm}|p{3cm}|p{7cm}|}
\hline
id & string & ID do tipo de equipamento \\ \hline
organizationId & string & Organização dona \\ \hline
name & string & Nome do tipo (Rádio, Colete Nível IIIA, etc.) \\ \hline
category & enum & \texttt{"EPI"}, \texttt{"COMMUNICATION"}, \texttt{"VEHICLE"}, \texttt{"ELECTRONIC"}, \texttt{"OTHER"} \\ \hline
requiresValidity & boolean & Se exige data de validade (EPIs) \\ \hline
requiresMaintenance & boolean & Se exige controle de manutenção \\ \hline
defaultValidityMonths & number/null & Prazo padrão de validade em meses \\ \hline
defaultMaintenanceIntervalKm & number/null & Intervalo padrão por quilometragem (veículos) \\ \hline
defaultMaintenanceIntervalDays & number/null & Intervalo padrão em dias \\ \hline
regulatoryTag & string/null & Referência a norma/regulamento interno \\ \hline
createdAt, createdByUserId, updatedAt, updatedByUserId, deletedAt, deletedByUserId, isDeleted & vários & Auditoria e soft delete \\ \hline
\end{longtable}

% -------- 4.8 equipments --------
\subsection{equipments}

Equipamentos individuais rastreáveis (cada rádio, colete, viatura).

\begin{longtable}{|p{4cm}|p{3cm}|p{7cm}|}
\hline
id & string & ID único do equipamento \\ \hline
organizationId & string & Organização dona \\ \hline
equipmentTypeId & string & Referência ao tipo de equipamento \\ \hline
serialNumber & string/null & Número de série do fabricante \\ \hline
assetTag & string & Etiqueta patrimonial interna (obrigatória) \\ \hline
status & enum & \texttt{"available"}, \texttt{"in\_use"}, \texttt{"in\_maintenance"}, \texttt{"decommissioned"}, \texttt{"lost"} \\ \hline
currentLocation & object & Localização atual lógica \\ \hline
\multicolumn{3}{|p{14cm}|}{
\textbf{Estrutura de currentLocation:}
\begin{itemize}
\item type: \texttt{"stock"}, \texttt{"post"}, \texttt{"operator"}, \texttt{"maintenanceProvider"}
\item refId: string/null
\end{itemize}
} \\ \hline
purchaseDate & datetime/null & Data de aquisição \\ \hline
warrantyExpiresAt & datetime/null & Fim de garantia \\ \hline
validUntil & datetime/null & Validade (EPIs) \\ \hline
contractId & string/null & Contrato dedicado (se aplicável) \\ \hline
notes & string/null & Observações \\ \hline
createdAt, createdByUserId, updatedAt, updatedByUserId, deletedAt, deletedByUserId, isDeleted & vários & Auditoria e soft delete \\ \hline
\end{longtable}

% -------- 4.9 assignments --------
\subsection{assignments}

Registros de movimentação de equipamentos (entrega, devolução, transferência).

\begin{longtable}{|p{4cm}|p{3cm}|p{7cm}|}
\hline
id & string & ID do registro de movimentação \\ \hline
organizationId & string & Organização dona \\ \hline
equipmentId & string & ID do equipamento envolvido \\ \hline
type & enum & \texttt{"checkout"}, \texttt{"checkin"}, \texttt{"transfer"} \\ \hline
fromLocation & object/null & Local anterior (pode ser null em checkout inicial) \\ \hline
toLocation & object & Local de destino \\ \hline
\multicolumn{3}{|p{14cm}|}{
\textbf{Estrutura básica de localização:}
\begin{itemize}
\item type: string
\item refId: string/null
\end{itemize}
} \\ \hline
operatorId & string/null & Operador específico (quando aplicável) \\ \hline
postId & string/null & Posto associado \\ \hline
termId & string/null & Termo digital vinculado \\ \hline
conditionBefore & enum/null & \texttt{"new"}, \texttt{"good"}, \texttt{"worn"}, \texttt{"damaged"} \\ \hline
conditionAfter & enum/null & Estado após devolução \\ \hline
notes & string/null & Observações \\ \hline
createdByUserId & string & Usuário que registrou \\ \hline
createdAt, updatedAt, updatedByUserId, deletedAt, deletedByUserId, isDeleted & vários & Auditoria e soft delete \\ \hline
\end{longtable}

% -------- 4.10 terms --------
\subsection{terms}

Termos digitais de responsabilidade relacionados a entregas e uso de equipamentos.

\begin{longtable}{|p{4cm}|p{3cm}|p{7cm}|}
\hline
id & string & ID do termo \\ \hline
organizationId & string & Organização dona \\ \hline
operatorId & string & Operador signatário \\ \hline
assignmentIds & array<string> & Lista de IDs de assignments cobertos \\ \hline
signedAt & datetime & Data/hora de assinatura \\ \hline
signedBy & string & Nome do operador no momento da assinatura (PII) \\ \hline
signatureMethod & enum & \texttt{"in\_app"}, \texttt{"tablet"}, \texttt{"external"} \\ \hline
ipAddress & string/null & IP do dispositivo (PII) \\ \hline
userAgent & string/null & User agent do dispositivo (PII) \\ \hline
pdfUrl & string/null & URL do PDF do termo \\ \hline
blockchainHash & string/null & Hash de integridade, se usado \\ \hline
createdAt, createdByUserId, updatedAt, updatedByUserId, deletedAt, deletedByUserId, isDeleted & vários & Auditoria e soft delete \\ \hline
\end{longtable}

% -------- 4.11 maintenanceOrders --------
\subsection{maintenanceOrders}

Ordens de manutenção preventiva ou corretiva dos equipamentos.

\begin{longtable}{|p{4cm}|p{3cm}|p{7cm}|}
\hline
id & string & ID da manutenção \\ \hline
organizationId & string & Organização dona \\ \hline
equipmentId & string & Equipamento associado \\ \hline
type & enum & \texttt{"preventive"}, \texttt{"corrective"} \\ \hline
status & enum & \texttt{"open"}, \texttt{"in\_progress"}, \texttt{"closed"}, \texttt{"cancelled"} \\ \hline
openedAt & datetime & Data de abertura \\ \hline
closedAt & datetime/null & Data de fechamento \\ \hline
openedByUserId & string & Usuário que abriu a ordem \\ \hline
providerName & string/null & Nome do prestador de serviço \\ \hline
cost & number/null & Custo da manutenção \\ \hline
description & string/null & Descrição detalhada \\ \hline
nextDueAt & datetime/null & Próxima manutenção programada \\ \hline
createdAt, createdByUserId, updatedAt, updatedByUserId, deletedAt, deletedByUserId, isDeleted & vários & Auditoria e soft delete \\ \hline
\end{longtable}

% -------- 4.12 alerts --------
\subsection{alerts}

Alertas automáticos do sistema (EPI vencendo, atraso de devolução, etc.).

\begin{longtable}{|p{4cm}|p{3cm}|p{7cm}|}
\hline
id & string & ID do alerta \\ \hline
organizationId & string & Organização dona \\ \hline
type & enum & \texttt{"epi\_expiry"}, \texttt{"late\_return"}, \texttt{"maintenance\_due"}, \texttt{"stock\_low"} \\ \hline
severity & enum & \texttt{"info"}, \texttt{"warning"}, \texttt{"critical"} \\ \hline
equipmentId & string/null & Equipamento relacionado (se houver) \\ \hline
operatorId & string/null & Operador relacionado (se houver) \\ \hline
contractId & string/null & Contrato relacionado (se houver) \\ \hline
message & string & Mensagem de alerta \\ \hline
createdAt & datetime & Data de criação do alerta \\ \hline
resolvedAt & datetime/null & Data de resolução \\ \hline
resolvedByUserId & string/null & Usuário que resolveu \\ \hline
deletedAt, deletedByUserId, isDeleted & vários & Controle de soft delete (se aplicado) \\ \hline
\end{longtable}

% -------- 4.13 auditLogs --------
\subsection{auditLogs}

Trilha de auditoria das ações relevantes do sistema.

\begin{longtable}{|p{4cm}|p{3cm}|p{7cm}|}
\hline
id & string & ID do log \\ \hline
organizationId & string & Organização dona \\ \hline
actorType & enum & \texttt{"user"}, \texttt{"system"} \\ \hline
actorId & string & ID do usuário ou identificador de sistema \\ \hline
action & string & Ação realizada (ex.: \texttt{"equipment.checkout"}) \\ \hline
entityType & string & Tipo de entidade (ex.: \texttt{"equipment"}) \\ \hline
entityId & string & ID da entidade afetada \\ \hline
before & object/null & Snapshot antes da ação (parcial ou completo) \\ \hline
after & object/null & Snapshot após a ação (parcial ou completo) \\ \hline
ipAddress & string/null & IP de origem (PII) \\ \hline
createdAt & datetime & Data/hora do registro \\ \hline
\end{longtable}

\noindent
\textbf{Observação}: por padrão, \texttt{auditLogs} não sofre updates ou deletes; \texttt{updatedAt}/\texttt{deletedAt} normalmente não são utilizados aqui, salvo políticas específicas.

% -------- 4.14 customFields --------
\subsection{customFields (opcional)}

Permite extensão de campos em entidades específicas por organização.

\begin{longtable}{|p{4cm}|p{3cm}|p{7cm}|}
\hline
id & string & ID do campo customizado \\ \hline
organizationId & string & Organização dona \\ \hline
targetCollection & string & Nome da coleção alvo (ex.: \texttt{"equipments"}) \\ \hline
fieldKey & string & Identificador interno (ex.: \texttt{"centroCusto"}) \\ \hline
label & string & Rótulo exibido em tela \\ \hline
type & enum & \texttt{"string"}, \texttt{"number"}, \texttt{"boolean"}, \texttt{"date"} \\ \hline
required & boolean & Indica se é obrigatório \\ \hline
createdAt, createdByUserId, updatedAt, updatedByUserId, deletedAt, deletedByUserId, isDeleted & vários & Auditoria e soft delete \\ \hline
\end{longtable}
\end{document}
