
\documentclass[12pt]{article}
\usepackage[utf8]{inputenc}
\usepackage{geometry}
\geometry{margin=2.5cm}
\usepackage{titlesec}
\usepackage{hyperref}
\hypersetup{colorlinks=true, linkcolor=blue, urlcolor=blue}

\title{OpSafe Domain Overview}
\author{}
\date{}

\begin{document}

\maketitle

\section{Visão Geral do Sistema}
O \textbf{OpSafe} é um sistema de gestão operacional para empresas de segurança privada, permitindo controle sobre operadores, clientes, postos de serviço, contratos, equipamentos, manutenção, alertas, auditoria e campos personalizados, tudo com arquitetura multi-organização.

\section{Estrutura de Domínio}
O domínio é dividido em três camadas principais:

\begin{itemize}
    \item \textbf{Configuração} -- entidades administrativas que estruturam o ambiente.
    \item \textbf{Operação} -- fluxo diário operacional.
    \item \textbf{Serviços Transversais} -- autenticação, auditoria e extensões.
\end{itemize}

\section{Camada de Configuração}

\subsection{Organizations}
Raiz do sistema, cada organização possui isolamento completo e define o espaço de dados.

\subsection{Users}
Usuários administrativos com papéis (admin, manager, viewer). Controlam autenticação e acesso ao sistema.

\subsection{Clients}
Clientes que recebem os serviços de segurança. Incluem dados como nome, documento, contato e endereço.

\subsection{Contracts}
Contratos firmados com os clientes. Relacionam-se a postos, SLA e vigência.

\subsection{Posts}
Postos de serviço onde operadores e equipamentos são alocados.

\subsection{EquipmentTypes}
Categorias e tipos de equipamentos, como EPIs, rádios, coletes etc.

\subsection{Equipments}
Equipamentos individualizados com status, numeração e características.

\subsection{Operators}
Colaboradores operacionais, com código identificador, documento, contato e turnos.

\subsection{CustomFields}
Campos personalizados criados pela organização para ampliar o modelo.

\section{Camada Operacional}

\subsection{Assignments}
Entrega e devolução de equipamentos. Controla ciclo operacional e vinculação a operadores e postos.

\subsection{Terms}
Termos digitais de responsabilidade assinados pelos operadores.

\subsection{MaintenanceOrders}
Ordens de manutenção que alteram o status do equipamento.

\subsection{Alerts}
Alertas operacionais e incidentes envolvendo operadores, equipamentos ou procedimentos.

\section{Camada Transversal}

\subsection{Auth}
Controle de autenticação via JWT, permissões e fluxo de convite.

\subsection{AuditLogs}
Registro de ações relevantes para rastreabilidade e compliance.

\section{Pilares do Domínio}
\begin{itemize}
    \item Multi-organização
    \item Controle operacional completo
    \item Compliance robusto
    \item Extensibilidade via campos customizados
\end{itemize}

\end{document}
