\documentclass[12pt,a4paper]{article}

\usepackage[utf8]{inputenc}
\usepackage[brazil]{babel}
\usepackage{geometry}
\usepackage{setspace}
\usepackage{titlesec}
\usepackage{longtable}

\geometry{margin=2.5cm}
\setstretch{1.25}

\titleformat{\section}{\Large\bfseries}{\thesection}{1em}{}
\titleformat{\subsection}{\large\bfseries}{\thesubsection}{1em}{}

\begin{document}

% ======================= CAPA =======================
\begin{titlepage}
    \centering
    {\Huge \textbf{OpSafe -- Projeto Lógico de Banco de Dados (MongoDB)}}\\[1.5cm]
    {\Large \textbf{Versão 1.0}}\\[0.5cm]
    {\large Documento Técnico Empresarial}\\[1cm]
    {\large Autor: Caio César Ponte}\\[0.3cm]
    {\large \textbf{Data: 15 de novembro de 2025}}\\[4cm]

    \vfill
    {\small Este documento descreve o projeto lógico do banco de dados MongoDB utilizado pela plataforma OpSafe, incluindo estrutura de databases, coleções, padrões de modelagem, índices e diretrizes de uso.}
\end{titlepage}

\tableofcontents
\newpage

% ======================= 1. INTRODUÇÃO =======================
\section{Introdução}

O OpSafe é uma plataforma SaaS orientada à gestão de equipamentos, operadores e contratos em empresas de segurança privada.  
O banco de dados principal utiliza MongoDB, no serviço gerenciado MongoDB Atlas.

Este documento registra o projeto lógico do banco, servindo de referência para:

\begin{itemize}
    \item desenvolvimento backend;
    \item testes e QA;
    \item governança de dados;
    \item evolução de schema e índices.
\end{itemize}

% ======================= 2. ESTRUTURA GERAL =======================
\section{Estrutura Geral do Cluster}

\subsection{Cluster e Databases}

O cluster MongoDB é provisionado no MongoDB Atlas.

São definidos, no mínimo, os seguintes databases:

\begin{itemize}
    \item \textbf{opsafe\_dev} -- ambiente de desenvolvimento;
    \item \textbf{opsafe\_stage} -- ambiente de homologação (futuro);
    \item \textbf{opsafe\_prod} -- ambiente de produção (futuro).
\end{itemize}

Neste documento, o foco é o database \textbf{opsafe\_dev}, que possui todas as coleções de negócio utilizadas pela aplicação.

\subsection{Lista de Coleções}

O database \textbf{opsafe\_dev} contém as seguintes coleções:

\begin{longtable}{|p{6cm}|p{8cm}|}
\hline
\textbf{Coleção} & \textbf{Descrição}\\ \hline
organizations & Organizações (empresas) que utilizam o OpSafe \\ \hline
users & Usuários internos da plataforma \\ \hline
operators & Operadores que recebem e utilizam equipamentos \\ \hline
clients & Clientes atendidos pelas organizações de segurança \\ \hline
contracts & Contratos firmados com os clientes \\ \hline
posts & Postos de serviço vinculados a contratos \\ \hline
equipmentTypes & Tipos de equipamentos (categorias) \\ \hline
equipments & Equipamentos individuais rastreáveis \\ \hline
assignments & Registros de movimentação de equipamentos \\ \hline
terms & Termos digitais de responsabilidade \\ \hline
maintenanceOrders & Ordens de manutenção \\ \hline
alerts & Alertas automáticos \\ \hline
auditLogs & Trilhas de auditoria \\ \hline
customFields & Campos personalizados por organização \\ \hline
\end{longtable}

% ======================= 3. PADRÕES DE MODELAGEM =======================
\section{Padrões de Modelagem}

\subsection{Multi-tenant por organizationId}

Todas as coleções de negócio, com exceção de \texttt{organizations}, utilizam o campo:

\begin{itemize}
    \item \textbf{organizationId} (string) -- identifica a empresa à qual o documento pertence.
\end{itemize}

Regras:

\begin{itemize}
    \item Toda consulta de aplicação deve filtrar por \texttt{organizationId};
    \item Não há compartilhamento de documentos entre organizações diferentes;
    \item Índices multi-tenant utilizam \texttt{organizationId} como primeiro campo.
\end{itemize}

\subsection{Auditoria e Soft Delete}

As coleções de negócio utilizam um padrão comum de auditoria:

\begin{itemize}
    \item \textbf{createdAt}, \textbf{createdByUserId};
    \item \textbf{updatedAt}, \textbf{updatedByUserId};
    \item \textbf{deletedAt}, \textbf{deletedByUserId};
    \item \textbf{isDeleted} (boolean).
\end{itemize}

Soft delete:

\begin{itemize}
    \item Quando \texttt{isDeleted = true}, o registro é considerado inativo;
    \item Consultas operacionais devem utilizar \texttt{isDeleted = false};
    \item \texttt{auditLogs} é a única coleção que, por padrão, não utiliza soft delete.
\end{itemize}

\subsection{Classificação de Dados (LGPD)}

De forma conceitual, os campos são classificados como:

\begin{itemize}
    \item \textbf{PII}: dados pessoais identificáveis (nome, e-mail, telefone, IP);
    \item \textbf{Sensíveis}: dados pessoais sensíveis (por exemplo, CPF, quando utilizado);
    \item \textbf{Business}: dados puramente operacionais.
\end{itemize}

Esta classificação guia decisões de minimização de dados, retenção e anonimização.

% ======================= 4. ÍNDICES PADRÃO =======================
\section{Índices Padrão por Coleção}

Esta seção registra os índices considerados mínimos para garantir desempenho adequado das consultas mais comuns.  
Todos os índices são definidos no database \textbf{opsafe\_dev} e deverão ser replicados em \textbf{opsafe\_prod}.

\subsection{organizations}

\begin{itemize}
    \item \texttt{\{ status: 1 \}}
    \item \texttt{\{ name: 1 \}}
    \item \texttt{\{ isDeleted: 1 \}}
\end{itemize}

\subsection{users}

\begin{itemize}
    \item \texttt{\{ organizationId: 1, email: 1 \}} (único);
    \item \texttt{\{ organizationId: 1, status: 1 \}};
    \item \texttt{\{ organizationId: 1, isDeleted: 1 \}}.
\end{itemize}

\subsection{operators}

\begin{itemize}
    \item \texttt{\{ organizationId: 1, status: 1 \}};
    \item \texttt{\{ organizationId: 1, identifierCode: 1 \}};
    \item \texttt{\{ organizationId: 1, isDeleted: 1 \}}.
\end{itemize}

\subsection{clients}

\begin{itemize}
    \item \texttt{\{ organizationId: 1, name: 1 \}};
    \item \texttt{\{ organizationId: 1, isDeleted: 1 \}}.
\end{itemize}

\subsection{contracts}

\begin{itemize}
    \item \texttt{\{ organizationId: 1, clientId: 1, status: 1 \}};
    \item \texttt{\{ organizationId: 1, isDeleted: 1 \}}.
\end{itemize}

\subsection{posts}

\begin{itemize}
    \item \texttt{\{ organizationId: 1, clientId: 1, status: 1 \}};
    \item \texttt{\{ organizationId: 1, isDeleted: 1 \}}.
\end{itemize}

\subsection{equipmentTypes}

\begin{itemize}
    \item \texttt{\{ organizationId: 1, category: 1 \}};
    \item \texttt{\{ organizationId: 1, isDeleted: 1 \}}.
\end{itemize}

\subsection{equipments}

\begin{itemize}
    \item \texttt{\{ organizationId: 1, status: 1 \}};
    \item \texttt{\{ organizationId: 1, equipmentTypeId: 1, status: 1 \}};
    \item \texttt{\{ organizationId: 1, assetTag: 1 \}} (único);
    \item \texttt{\{ organizationId: 1, validUntil: 1, status: 1 \}};
    \item \texttt{\{ organizationId: 1, isDeleted: 1 \}}.
\end{itemize}

\subsection{assignments}

\begin{itemize}
    \item \texttt{\{ organizationId: 1, equipmentId: 1, createdAt: -1 \}};
    \item \texttt{\{ organizationId: 1, operatorId: 1, createdAt: -1 \}};
    \item \texttt{\{ organizationId: 1, isDeleted: 1 \}}.
\end{itemize}

\subsection{terms}

\begin{itemize}
    \item \texttt{\{ organizationId: 1, operatorId: 1, signedAt: -1 \}};
    \item \texttt{\{ organizationId: 1, createdAt: -1 \}};
    \item \texttt{\{ organizationId: 1, isDeleted: 1 \}}.
\end{itemize}

\subsection{maintenanceOrders}

\begin{itemize}
    \item \texttt{\{ organizationId: 1, equipmentId: 1, status: 1 \}};
    \item \texttt{\{ organizationId: 1, status: 1, openedAt: -1 \}};
    \item \texttt{\{ organizationId: 1, isDeleted: 1 \}}.
\end{itemize}

\subsection{alerts}

\begin{itemize}
    \item \texttt{\{ organizationId: 1, type: 1, severity: 1, createdAt: -1 \}};
    \item \texttt{\{ organizationId: 1, isDeleted: 1 \}}.
\end{itemize}

\subsection{auditLogs}

\begin{itemize}
    \item \texttt{\{ organizationId: 1, entityType: 1, entityId: 1, createdAt: -1 \}};
    \item \texttt{\{ organizationId: 1, createdAt: -1 \}}.
\end{itemize}

\subsection{customFields}

\begin{itemize}
    \item \texttt{\{ organizationId: 1, targetCollection: 1 \}};
    \item \texttt{\{ organizationId: 1, targetCollection: 1, fieldKey: 1 \}} (único);
    \item \texttt{\{ organizationId: 1, isDeleted: 1 \}}.
\end{itemize}

% ======================= 5. DIRETRIZES DE USO =======================
\section{Diretrizes de Uso no Backend}

\begin{itemize}
    \item Toda camada de repositório deve aplicar, por padrão, filtros de \texttt{organizationId} e \texttt{isDeleted = false};
    \item Operações de exclusão lógica devem preencher \texttt{deletedAt}, \texttt{deletedByUserId} e \texttt{isDeleted = true};
    \item A coleção \texttt{auditLogs} não deve sofrer updates ou deletes, salvo regras excepcionais de retenção definida em política separada;
    \item Novos índices devem ser documentados e replicados entre ambientes.
\end{itemize}

\end{document}
